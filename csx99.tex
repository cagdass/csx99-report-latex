\documentclass[10pt]{article}
\usepackage[utf8]{inputenc}
\usepackage{graphicx}
\usepackage{indentfirst}
\usepackage{etoolbox}
\patchcmd{\thebibliography}{\section*}{\section}{}{}
\usepackage{subfig}
\graphicspath{ {images/} }
\usepackage{titling}
\usepackage{mathtools}
\usepackage{rotating}
\usepackage{tikz}

\setlength{\droptitle}{-10em} % We don't want to waste half the page.

\title{BILKENT UNIVERSITY \\ ~\\ ENGINEERING FACULTY \\ ~\\
DEPARTMENT OF COMPUTER ENGINEERING \\ ~\\ ~\\}

% Please change the course code, name and ID number. No need to add anything following "Performed at".
\author{CSx99 \\ SUMMER TRAINING REPORT \\ ~\\ NAMEHERE \\ ID NUMBER \\ ~\\ Performed at: }

\begin{document}

\begin{titlingpage} % Starts the title page.
    \begin{center}
        \begin{large}
            \textbf{\thetitle} % Put title here.
        \end{large}

        \includegraphics[scale=1.3]{bilkent_logo.png}\\ % Put the logo here.

        \begin{LARGE}
            \textbf{\theauthor} % Author here. Not exactly the author though.
        \end{LARGE}

        % If the company logo bears a different file name, change the file name or change the path on the following line.
        \includegraphics[scale=0.3]{company_logo.png} % Put the company logo here.

        % Please change the company name and dates accordingly.
        \begin{Large}
            \textbf{\\ COMPANY ~\\}
            \textbf{\\ START DATE - END DATE}
        \end{Large}
    \end{center}
\end{titlingpage}

\tableofcontents

\newpage

\section{Abstract}

Content.

\newpage

\section{Introduction}

% Have an introductory section that will make a smooth beginning to the document.
% In the introduction section include the following:
% * The name of the company and department where you have done your summer training,
% the main focus area of the company, and your motivation for choosing this company as
% the place for your summer training.
% * Brief summary of the work you have done, the motivation behind it, and the significance
% of the work that you have done in the overall project.
% * Explanation of the organization of the rest of the report.

Content.

\section{Company Information}

% Have a section providing information about the company and
% department where you did your training, its hardware/software systems and resources, its focus
% and project area, its organization, etc. Do not write in too much detail. The name, address,
% telephone number, email address, and information about the education of your supervisor must
% be given (including the name of the university and department from which he/she graduated, and
% the year of graduation). The required format for this section is given in Summer Training Report
% Template file. (http://www.cs.bilkent.edu.tr/SummerTraining/Report%20Template.dotx)

Content.

\subsection{About the Company}

Content.

\subsection{About the Department}

Content.

\subsection{About the Hardware and Software Systems}

Content.

\\ \textbf{Hardware} \\

Content.

\\ \textbf{Software} \\

Content.

\subsection{Team and Supervisor Information}

Content.

% The supervisor’s name and job title, along with his or her university and department and year of graduation must be stated here.
% Please change the following information accordingly.

\textbf{My Supervisor:}
\begin{itemize}
    \item{\textbf{Name: } }
    \item{\textbf{Title: } }
    \item{\textbf{Education:} }
    \begin{itemize}
        \item{BS in Computer Engineering, }
        \item{MS in Computer Science, }
    \end{itemize}
    \item{\textbf{E-Mail: } }
\end{itemize}

\section{Work Done}

% This is the most important part of your report. The number of sub-sections in this
% part, their titles, and their contents depend on the work that you have done and the information
% you would like to provide. This part should include at least the following:
% o Information about the main project, if the work you have done is part of such a
% project.
% o The significance of the work you have done.
% o The motivation behind the particular work that you have done and why it is
% required.
% o Detailed description of the work done, including for example:
%  - The algorithms/pseudo-code developed.
%  - Hardware/software environment used.
%  - Software tools used.
%  - Design methods used and learned.
%  - Testing methods and tools used and learned.
%  - Project management methods and processes followed or observed.
%  - Any engineering standards that are followed or observed.
%  - Design, development, documentation and testing participated in or
%  observed.
%  - Any configuration and/or maintenance tasks performed.
% o Detailed description of your own contribution and clearly identification of the
% distinctions from others’ work.
% o When writing this section, do not forget that the reader may not be familiar with
% the topic of the work that you have done. Therefore, explaining too much is better
% than not enough. But long code sections should be in the Appendix, not in the
% body of the report

Content.

\section{Performance and Outcomes}
% You should also write and discuss the outcomes of your summer
% training by writing the following 9 sections in your report, each corresponding to one of the 9
% Evaluation criteria:

Content.

\subsection{Applying Knowledge and Skills Learned at Bilkent}

% A section titled: Applying Knowledge and Skills Learned at Bilkent in which
% you explain in detail what knowledge and skills learned in school you were able
% to apply to real-world problems during your summer training, and specifically
% where and how the knowledge or skills were useful

Content.

\subsection{Solving Engineering Problems}

% A section titled: Solving Engineering Problems in which you explain in detail
% the engineering problems related to computer systems and applications that you
% solved.

Content.

\subsection{Team Work}

% A section titled: Team Work in which you explain in detail the teamwork you
% were involved in during the summer training, including (for each team you
% participated in) the team role or function of each team member, the training in
% their background and current work area, and some information about the team
% dynamics as you worked together. You should clearly explain how you related to
% the others on the team. If you were not involved in a formal team, the definition
% of the term could be interpreted loosely to mean working together with others on
% a shared task.

Content.

\subsection{Multi-Disciplinary Work}

% A section titled: Multi-disciplinary Work in which you explain how you
% worked with team-mates from other disciplines (multi-disciplinary work). If you
% did not engage in a multi-disciplinary work or team, indicate this in this section
% clearly.

Content.

\subsection{Professional and Ethical Issues}

% A section titled: Professional and Ethical Issues in which you explain in detail
% which professional issues and work-related ethical issues you saw or became
% aware of during your summer training, and how the issue was handled or
% managed at the company.

Content.

\subsection{Impact of Engineering Solutions}

% A section titled: Impact of Engineering Solutions in which you explain
% specifically what you learned or understood about the economic, environmental,
% societal and global impact of the engineering solutions in the projects developed
% at the company.

Content. \cite{something}

\subsection{Locating Sources and Self-Learning}

% A section titled: Locating Sources and Self-Learning in which you explain the
% self-learning that you did during your summer training. You should mention any
% sources that you located and how you found them (this would include Web sites,
% books, journals, experts, etc), and what part of your summer training task you
% needed them for. Also, mention any that you made regular use of, and any that
% you are continuing to use.

Content.

\subsection{Knowledge about Contemporary Issues}

% A section titled: Knowledge about Contemporary Issues where you write about
% the contemporary issues that are related with computer engineering, as you
% understand them from, and related to, your summer training.

Content.

\subsection{Using New Tools and Technologies}

% A section titled: Using New Tools and Technologies in which you explain in
% detail any new tools or technologies that you encountered and used during your
% summer training, how you learned to use them, and what level of proficiency you
% came to by the end of your summer training.

Content.

\section{Conclusion}

Content. \cite{another}

\section{References}

Content.

\begin{thebibliography}{99}

\bibitem{something} Some citation.

\bibitem{another} Some other citation.

\end{thebibliography}

\section{Appendix}

Content.

\end{document}
